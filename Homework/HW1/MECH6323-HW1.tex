\documentclass[letter]{article}
\renewcommand{\baselinestretch}{1.25}

\usepackage[margin=1in]{geometry}
\usepackage{physics}
\usepackage{amsmath, mathtools}
\numberwithin{equation}{section}
\usepackage{amssymb}
\usepackage{graphicx}
\usepackage{hyperref}
\usepackage{empheq}

% MATLAB Formating Code
\usepackage[numbered,framed]{matlab-prettifier}
\lstset{style=Matlab-editor,columns=fullflexible}
\renewcommand{\lstlistingname}{Script}
\newcommand{\scriptname}{\lstlistingname}

\allowdisplaybreaks

%opening
\title{MECH 6323 - HW 1}
\author{Jonas Wagner}
\date{2022, January 24}

\begin{document}	

\maketitle

% \tableofcontents

%----------------------------------------------------------------------------
% \newpage
% \section{Problem 1}

When controlling a dynamical system, a designer must alway make decisions that will have trade-offs.
One of these important thing to weigh is how effective a system is at reacting to disturbances, so the robustness of the control design is very important.
As mentioned in the original lecture, disturbances can range from design inaccuracies, process or measurement noise, or external physical disturbances.

Engineering ethics plays an important role in all of engineering.
Robustness trade-offs vs performance or price specifically can have many ethical considerations for whether sacrificing safety or security gained from robustness is ethical.

The topics in this course that would be most beneficial to my research are the methods used to prove robustness to different disturbances.
One of my research topics (CPS Security) is determining the worse case scenario that malicious disturbances can have on a system without being detected.
The concern of security aligns closely to robustness and so methods and proofs are also similar.

Lectures themselves, considering the limitations of being online, are okay.
Compared to lectures in Nonlinear Systems I think there were a few things that are great and also things that could be improved:
\begin{itemize}
	\item Good Things
	\begin{itemize}
		\item Recording lectures and posting notes after the fact are good to refer back to
		\item Involving us in the lecture and being open for questions about the material
				(even though a lot of people don't actually get involved)
	\end{itemize}
	\item Potential Improvements
	\begin{itemize}
		\item Notes based on topic as opposed to just per lecture
		\item Pre-lecture notes or real-time note sharing w/ OneNote or something 
				(often the switching to a new pages is fast and can't write down what you put on the page)
		\item Not a lecture thing but: 
		\begin{itemize}
			\item Set Midterm exam date
			\item Provide a tentative timeline for the semester
		\end{itemize}
	\end{itemize}
\end{itemize}


% \newpage
% \appendix
% \section{MATLAB Code:}\label{apx:matlab}
% All code I write in this course can be found on my GitHub repository:\\
% \href{https://github.com/jonaswagner2826/MECH6313}{https://github.com/jonaswagner2826/MECH6313}


\end{document}
